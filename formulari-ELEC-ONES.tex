\documentclass[10pt,twocolumn]{article}
\usepackage{amsmath}
\usepackage{siunitx}
\usepackage{bm}
\usepackage[compact]{titlesec}
\usepackage[catalan]{babel}

\allowdisplaybreaks

\titleformat*{\section}{\large \bfseries}
\titleformat*{\subsection}{\normalfont\bfseries}
\titleformat*{\subsubsection}{\normalfont}

\begin{document}

\section{Electrònica}
\subsection{Díodes}
\begin{gather*}
V_p - V_n \geq V_\gamma \implies \text{PD} \\
\text{no PD} \implies \text{PI} \\
\text{PD} \implies I \neq 0 \\
\text{PI} \implies I = 0 \\
P_{cons} = \Delta V I \\
\text{PI} \quad \text{i} \quad \Delta V_{Z \text{co}} \geq V_Z \implies \text{regió Zener}
\end{gather*}
\subsection{NMOS}
\begin{gather*}
V_{GS} = V_G - V_S \qquad V_{DS} = V_D - V_S \\
V_{GS} \leq V_T \implies \text{Tall (OFF)} \implies I_D = 0 \\
V_{GS} > V_T \implies \text{Canal (ON)} \implies I_D \neq 0  \\
V_{DS} < V_{GS} - V_T \implies \text{Regió òhmica} \\
V_{DS} < V_{GS} - V_T \iff V_{GD} > V_T \\
V_{DS} > V_{GS} - V_T \implies \text{Regió de saturació} \\
V_{DS} > V_{GS} - V_T \iff V_{GD} < V_T
\end{gather*}
\subsection{PMOS}
\begin{gather*}
    V_{GS} \geq V_T \implies \text{Tall (OFF)} \\
    V_{GS} < V_T \implies \text{Canal (ON)} \\
    V_{DS} > V_{GS} - V_T \implies \text{Regió òhmica} \\
    V_{DS} < V_{GS} - V_T \implies \text{Regió de saturació}
\end{gather*}
\subsection{Shockley}
\begin{gather*}
    \text{Òhmica} \implies I_D = \beta \left(\left(V_{GS} - V_T\right)V_{DS} - \frac{V_{DS}^{2}}{2}\right) \\
    \text{Saturació} \implies I_D = \frac{\beta}{2} \left(V_{GS} - V_T\right)^2
\end{gather*}
\subsection{CMOS}
\begin{gather*}
    t_\text{PHL} = 1,7\frac{C}{\beta_N V_{DD}} \\
    t_\text{PLH} = 1,7\frac{C}{\beta_P V_{DD}} \\
    t_\text{P} = \frac{t_\text{PHL} + t_\text{PLH}}{2} \\
    \mathcal{U} = \frac{1}{2} C V_{DD}^2 \\
    P = fCV_{DD}^2 \\
    \text{DP} = Pt_\text{P}
\end{gather*}

\section{Ones}
\subsection{Equació d'ona}
\begin{gather*}
    \text{Funció d'ona} \equiv \psi \left(x, t\right) \\
    \frac{\partial^2 \psi}{\partial x^2} = \frac{1}{v^2}\frac{\partial^2\psi}{\partial t} \implies \psi\left(x, t\right) = f\left(x \pm vt\right) \\
    f\left(x \bm{+} vt\right) \implies \text{mov cap a la dreta} \\
    f\left(x \bm{-} vt\right) \implies \text{mov cap a l'equerra}
\end{gather*}
\subsection{Ones harmòniques}
\begin{gather*}
    \psi\left(x, t\right) = A\sin\left(k\left(x - vt\right) + \delta \right) \\
    f = \frac{1}{T} \\
    \omega = \frac{2\pi}{T} = 2\pi f \\
    k = \frac{2\pi}{\lambda} \\
    v = \frac{\omega}{k} \\
    v = \lambda f \\
    I = \frac{P}{S} \tag{\si{W/m^2}}
\end{gather*}
\subsection{Ones electromatnètiques}
\begin{gather*}
    c = v = \frac{1}{\sqrt{\varepsilon_0 \mu_0}} = \SI{3e8}{\meter/\second} \\
    E = cB \\
    \vec{E} \perp \vec{B} \\
    \vec{u} = \frac{\vec{E}\times\vec{B}}{|\vec{E}\times\vec{B}|} \implies \ref{eq:1} \quad \text{i} \quad \ref{eq:2}\\
    \vec{B} = \frac{1}{c}\vec{u}\times\vec{E}\tag{*}\label{eq:1} \\
    \vec{E} = c\left(\vec{B}\times\vec{u}\right) \tag{**}\label{eq:2} \\
    \eta_E = \frac{1}{2}\varepsilon_0 E^2 \qquad \eta_B = \frac{1}{2} \frac{B^2}{\mu_0} \tag{\si{J/m^3}}\\
    \eta_\text{ona} = \varepsilon_0E^2 = \frac{B^2}{\mu_0} = \frac{\left|E\right|\left|B\right|}{\mu_0 c} \tag{\si{J/m^3}} \\
    \text{Vec. de Poynting: } \vec{S} = \frac{\vec{E}\times\vec{B}}{\mu_0} \tag{\si{W/m^2}} \\
    I = \left<\left|\vec{S}\right|\right> = c\left<\eta_\text{ona}\right> = \frac{E^2}{2c\mu_0} = \frac{B^2 c}{2\mu_0}
\end{gather*}
\subsection{Polarització}
\begin{gather*}
    I_0 = \frac{E^2}{2c\mu_0} \\
    I_1 = \frac{I_0}{2} \\
    I_{n+1} = I_n \cos^2 \theta
\end{gather*}
\subsection{Reflexió i refracció}
\begin{gather*}
n = \frac{c}{v} \implies n \geq 1 \\
\lambda' = \frac{v}{f} = \frac{c}{f}\frac{1}{n} = \frac{\lambda}{n} \\
n_1\sin \alpha_1 = n_2\sin \alpha_2 \\
n_1 < n_2 \implies \alpha_2 < \alpha_1 \\
n_2 < n_1 \implies \alpha_1 < \alpha_2 \\
\alpha_1 = \alpha_c \implies \alpha_2 = \ang{90} \\
n_1\sin \alpha_c = n_2\sin \ang{90} \implies \sin \alpha_c = \frac{n_2}{n_1} \\
\alpha_1 \geq \alpha_c \implies \text{Reflexió total interna}
\end{gather*}
\subsection{Fibres òptiques}
\begin{gather*}
    n_\text{recobriment} < n_\text{nucli} \\
    \sin \alpha_c = \frac{n_r}{n_c} \\
    \sin \alpha_0 = n_n \sin\left(\ang{90}-\alpha_c\right)
\end{gather*}

\end{document}
